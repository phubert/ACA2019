\documentclass[10pt]{beamer}
\mode<presentation>
\setbeamertemplate{navigation symbols}{}

\usepackage[francais]{babel}
\usepackage[utf8]{inputenc}
\usepackage{geometry,listings} 
\usepackage{graphicx}
\usepackage{float}
\usepackage{mathrsfs} 
\usepackage{color,xcolor,moreverb}	
\usepackage[center]{caption}
\usepackage{amsfonts,amsmath,amssymb,mathtools}
\usepackage[mathcal]{euscript}
\usepackage{tikz,genyoungtabtikz}
\usepackage{etex}
\usepackage{ulem} 
\usepackage{multicol}
\usepackage{multirow}
\usepackage{svg}
\usepackage{bbm, bm}

\newcommand{\blue}[1]{{\color[rgb]{0,0.4,1}{#1}}}
\newcommand{\red}[1]{{\color{red}{#1}}}
\newcommand{\green}[1]{{\color[rgb]{0,0.55,0.25}{#1}}}
\newcommand{\purple}[1]{{\color[rgb]{.7,.2,1}{#1}}}
\newcommand{\orange}[1]{{\color[rgb]{1,.35,0}{#1}}}

\definecolor{purple}{rgb}{.6,.2,1}
\definecolor{blue}{rgb}{0,0.6,1}
\definecolor{green2}{rgb}{0,0.55,0.25}

\usetheme{Boadilla}

\setbeamertemplate{blocks} [rounded] [shadow=true]
\setbeamersize{text margin left=8mm,text margin right=8mm}

\newenvironment{myblock}[3]{%
	\setbeamercolor{block body}{#2}
	\setbeamercolor{block title}{#3}
	\begin{block}{#1}}{\end{block}}

\setbeamertemplate{footline}
{
	\leavevmode%
	%\hbox{%
		\null \hfill
		\begin{beamercolorbox}[wd=.2\paperwidth,ht=2.25ex,dp=1ex,right]{title in head/foot}%
			\insertframenumber{} / \inserttotalframenumber\hspace*{1ex} \hspace*{.5cm}
		\end{beamercolorbox}%}%
		\vskip0pt%
}


\newcommand{\diagramme}{\YFrench \Yboxdim{12pt}\yng}
\newcommand{\minidiagramme}{\YFrench \Yboxdim{6pt}\yng}
\newcommand{\tableau}{\YFrench \Yboxdim{12pt}\young}
\newcommand{\bigtableau}{\YFrench \Yboxdim{28pt}\young}
%\newcommand{\diagramme}{\Yboxdim{10pt}\yng}  % notation anglaise
%\newcommand{\tableau}{\Yboxdim{10pt}\young}  % notation anglaise

\DeclareMathOperator{\Enk}{\mathcal{E}_n^{\langle k\rangle}}
\DeclareMathOperator{\Mmuk}{\mathcal{M}_{\mu}^{\langle k\rangle}}
\DeclareMathOperator{\sgn}{sgn}
\DeclareMathOperator{\Harm}{Harm}
\DeclareMathOperator{\GLr}{GL_r}
\DeclareMathOperator{\GL}{GL}
\DeclareMathOperator{\Sn}{\mathbb{S}_n}
\DeclareMathOperator{\Sym}{\mathbb{S}}

\newcommand{\QQ}{\mathbb{Q}}
\newcommand{\NN}{\mathbb{N}}
\newcommand{\CC}{\mathbb{C}}

\newcommand{\Ek}[2]{\mathcal{E}_#1^{\langle #2\rangle}}


%  Contenu de la page de titre 
\title[]{Interactive tutorials, an example on symmetric	functions}
\author{Pauline Hubert \\ with Mélodie Lapointe}
\institute{Université du Québec à Montréal \\ (UQAM)}
\date[]{16 juillet 2019}


%-----------------------------------------
%    Debut document   
% ----------------------------------------
\begin{document}

% Première slide
\begin{frame}
\titlepage

\begin{center}	
ACA 2019 \\
Montreal
\end{center}

\end{frame}

% Plan
%\begin{frame}\frametitle{Plan}
%\tableofcontents[hideothersubsections]
%\end{frame} 

\section{Intro}

%\begin{frame}\frametitle{}
%
%\green{green} \blue{blue} \purple{purple} \red{red} \orange{orange}
%
%\vspace{.5cm} 
%
%\begin{myblock}{purple}{}{fg=black, bg=blue!50!red!35!white}
%	Stuff
%\end{myblock}
%
%\begin{myblock}{green}{}{fg=black, bg=blue!30!green!50!white}
%	Stuff
%\end{myblock}
%
%\begin{myblock}{red}{}{fg=black, bg=red!40!white}
%	Stuff
%\end{myblock}
%
%\begin{myblock}{blue}{}{fg=black, bg=green!30!blue!40!}
%	Stuff
%\end{myblock}
%
%\begin{myblock}{orange}{}{fg=black, bg=orange!80!red!60!}
%	Stuff
%\end{myblock}
%
%\end{frame}

%\begin{frame}\frametitle{}
%\Large
%\begin{center}
%	\red{Computational Algebra} 
%	\newline
%	
%	\hspace{3cm}{\normalsize\orange{python}}
%	
%	\hspace{-1cm}\blue{SageMath}
%	\newline
%	
%	\hspace{3cm} \green{symmetric functions} 
%	\newline
%	\vspace{.3cm}
%	
%	\hspace{-1.5cm}
%	{\normalsize \purple{interactive tutorials}}
%	
%%	\hspace{4cm}
%%	{\normalsize \orange{better}}
%\end{center}	
%\end{frame}}

\section{Sage}

\begin{frame}\frametitle{\blue{SageMath}}

\vspace{-.8cm}

\begin{figure}[h]
	\raggedleft
	\includegraphics[scale=.25]{logo.png}
\end{figure}

\vspace{1cm}

\begin{itemize}
	\item Free open source mathematical software
	\item Started by William Stein in 2005
	\item Built on top of many open source softwares (GAP, Maxima, R, ...)
	\item In Python 
	\item Hundreds of worldwide contributors
	\item Can be installed on Linux, Mac or Windows or used online (CoCalc.com)
	\item With Jupyter notebook 
\end{itemize}

\begin{figure}[h]
	\centering
	\includegraphics[scale=.05]{jupyter_logo.png}
	\includegraphics[scale=.1]{python.png}
\end{figure}

\end{frame}

\begin{frame}\frametitle{\blue{SageMath}}

\vspace{-.4cm}

\begin{figure}[h]
	\centering
	\includegraphics[scale=.26]{jupyter.png}
\end{figure}

\end{frame}

\section{Tutorial Symmetric Functions}

\begin{frame}[fragile]\frametitle{\purple{Tutorial on Symmetric Functions}}

Two reference pages for symmetric functions in Sage. \newline 
\begin{enumerate}
	\item Documentation : \\
	\url{http://doc.sagemath.org/html/en/reference/combinat/sage/combinat/sf/sfa.html} \newline
	 
	\item Previous tutorial : \\
	\url{http://doc.sagemath.org/html/en/reference/combinat/sage/combinat/sf/sf.html#sage.combinat.sf.sf.SymmetricFunctions} \newline
\end{enumerate}

\end{frame}

\begin{frame}\frametitle{\purple{Tutorial on Symmetric Functions}}

Why a new tutorial on symmetric functions ?  \newline

\begin{itemize}
	\item More maths and more definitions : accessible with few knowledge on symmetric functions
	\item Add exercises in addition to examples
	\item Add interactivity \newline
\end{itemize}

\textbf{The new tutorial:} \newline

\centering
\url{https://more-sagemath-tutorials.readthedocs.io/en/latest/tutorial-symmetric-functions.html}

\end{frame}

\section{An exercise}

\begin{frame}\frametitle{\green{Discriminant of a polynomial using symmetric functions}}

Let see an interactive exercise with Jupyter notebook using symmetric functions. 

Experimented during Sage days

\end{frame}


\section{Conclusion}

\begin{frame}\frametitle{\orange{Conclusion}}

\begin{itemize}
	\item The new tutorial could still be better : more definitions, exercises, etc.
	\item Compromise between a complete but not too long tutorial.\newline
	
	\item A tool to learn/teach mathematics
	\item Learn, experiment and practice in the same environment
	\item Can be use for research : experiments, take notes, write theorem, interactive proofs and interactive examples at the same place
	\item Interactive slides in Jupyter from a worksheet
	\item Access notebooks online with Binder 
\end{itemize}

\end{frame}

\begin{frame}
\Huge
\begin{figure}[h]
	\bigtableau({\purple{h}}{\red{a}}{\orange{n}}{\green{k}},{\blue{T}})\hspace*{0cm} \bigtableau({\red{o}}{\orange{u}},{\purple{Y}})
\end{figure}
\end{frame}


\newcounter{lastframe}
\setcounter{lastframe}{\insertframenumber}
\setcounter{framenumber}{\thelastframe}

\end{document}
