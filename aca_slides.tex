\documentclass[10pt]{beamer}
\mode<presentation>
\setbeamertemplate{navigation symbols}{}

\usepackage[francais]{babel}
\usepackage[utf8]{inputenc}
\usepackage{geometry,listings} 
\usepackage{graphicx}
\usepackage{float}
\usepackage{mathrsfs} 
\usepackage{color,xcolor,moreverb}	
\usepackage[center]{caption}
\usepackage{amsfonts,amsmath,amssymb,mathtools}
\usepackage[mathcal]{euscript}
\usepackage{tikz,genyoungtabtikz}
\usepackage{etex}
\usepackage{ulem} 
\usepackage{multicol}
\usepackage{multirow}
\usepackage{svg}
\usepackage{bbm, bm}

\newcommand{\blue}[1]{{\color[rgb]{0,0.4,1}{#1}}}
\newcommand{\red}[1]{{\color{red}{#1}}}
\newcommand{\green}[1]{{\color[rgb]{0,0.55,0.25}{#1}}}
\newcommand{\purple}[1]{{\color[rgb]{.7,.2,1}{#1}}}
\newcommand{\orange}[1]{{\color[rgb]{1,.35,0}{#1}}}

\definecolor{purple}{rgb}{.6,.2,1}
\definecolor{blue}{rgb}{0,0.6,1}
\definecolor{green2}{rgb}{0,0.55,0.25}

\usetheme{Boadilla}

\setbeamertemplate{blocks} [rounded] [shadow=true]
\setbeamersize{text margin left=8mm,text margin right=8mm}

\newenvironment{myblock}[3]{%
	\setbeamercolor{block body}{#2}
	\setbeamercolor{block title}{#3}
	\begin{block}{#1}}{\end{block}}

\setbeamertemplate{footline}
{
	\leavevmode%
	%\hbox{%
		\null \hfill
		\begin{beamercolorbox}[wd=.2\paperwidth,ht=2.25ex,dp=1ex,right]{title in head/foot}%
			\insertframenumber{} / \inserttotalframenumber\hspace*{1ex} \hspace*{.5cm}
		\end{beamercolorbox}%}%
		\vskip0pt%
}


\newcommand{\diagramme}{\YFrench \Yboxdim{12pt}\yng}
\newcommand{\minidiagramme}{\YFrench \Yboxdim{6pt}\yng}
\newcommand{\tableau}{\YFrench \Yboxdim{12pt}\young}
\newcommand{\bigtableau}{\YFrench \Yboxdim{28pt}\young}
%\newcommand{\diagramme}{\Yboxdim{10pt}\yng}  % notation anglaise
%\newcommand{\tableau}{\Yboxdim{10pt}\young}  % notation anglaise

\DeclareMathOperator{\Enk}{\mathcal{E}_n^{\langle k\rangle}}
\DeclareMathOperator{\Mmuk}{\mathcal{M}_{\mu}^{\langle k\rangle}}
\DeclareMathOperator{\sgn}{sgn}
\DeclareMathOperator{\Harm}{Harm}
\DeclareMathOperator{\GLr}{GL_r}
\DeclareMathOperator{\GL}{GL}
\DeclareMathOperator{\Sn}{\mathbb{S}_n}
\DeclareMathOperator{\Sym}{\mathbb{S}}

\newcommand{\QQ}{\mathbb{Q}}
\newcommand{\NN}{\mathbb{N}}
\newcommand{\CC}{\mathbb{C}}

\newcommand{\Ek}[2]{\mathcal{E}_#1^{\langle #2\rangle}}


%  Contenu de la page de titre 
\title[]{Interactive tutorials, an example on symmetric	functions}
\author{Pauline Hubert \\ Mélodie Lapointe}
\institute{Université du Québec à Montréal \\ (UQAM)}
\date[]{16 juillet 2019}


%-----------------------------------------
%    Debut document   
% ----------------------------------------
\begin{document}

% Première slide
\begin{frame}
\titlepage

\begin{center}	
ACA 2019 \\
Montreal
\end{center}

\end{frame}

% Plan
%\begin{frame}\frametitle{Plan}
%\tableofcontents[hideothersubsections]
%\end{frame} 

\section{Intro}

%\begin{frame}\frametitle{}
%
%\green{green} \blue{blue} \purple{purple} \red{red} \orange{orange}
%
%\vspace{.5cm} 
%
%\begin{myblock}{purple}{}{fg=black, bg=blue!50!red!35!white}
%	Stuff
%\end{myblock}
%
%\begin{myblock}{green}{}{fg=black, bg=blue!30!green!50!white}
%	Stuff
%\end{myblock}
%
%\begin{myblock}{red}{}{fg=black, bg=red!40!white}
%	Stuff
%\end{myblock}
%
%\begin{myblock}{blue}{}{fg=black, bg=green!30!blue!40!}
%	Stuff
%\end{myblock}
%
%\begin{myblock}{orange}{}{fg=black, bg=orange!80!red!60!}
%	Stuff
%\end{myblock}
%
%\end{frame}

\begin{frame}\frametitle{}
\Large
\begin{center}
	\red{ACA} 
	\newline
	
	\hspace{2cm}{\normalsize\blue{python}}
	
	\blue{SageMath}
	\newline
	
	\hspace{2cm} \purple{symmetric functions} 
	\newline
	\vspace{.3cm}
	
	\hspace{-2cm}
	{\normalsize \orange{interactive tutorials}}
	\newline
	
	\hspace{4cm}
	{\normalsize \green{better}}
\end{center}	
\end{frame}

\section{Sage}

\begin{frame}\frametitle{\blue{Sage}}

\end{frame}

\newcounter{lastframe}
\setcounter{lastframe}{\insertframenumber}
\setcounter{framenumber}{\thelastframe}

\end{document}
